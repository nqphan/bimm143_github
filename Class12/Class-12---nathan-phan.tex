% Options for packages loaded elsewhere
% Options for packages loaded elsewhere
\PassOptionsToPackage{unicode}{hyperref}
\PassOptionsToPackage{hyphens}{url}
\PassOptionsToPackage{dvipsnames,svgnames,x11names}{xcolor}
%
\documentclass[
  letterpaper,
  DIV=11,
  numbers=noendperiod]{scrartcl}
\usepackage{xcolor}
\usepackage{amsmath,amssymb}
\setcounter{secnumdepth}{-\maxdimen} % remove section numbering
\usepackage{iftex}
\ifPDFTeX
  \usepackage[T1]{fontenc}
  \usepackage[utf8]{inputenc}
  \usepackage{textcomp} % provide euro and other symbols
\else % if luatex or xetex
  \usepackage{unicode-math} % this also loads fontspec
  \defaultfontfeatures{Scale=MatchLowercase}
  \defaultfontfeatures[\rmfamily]{Ligatures=TeX,Scale=1}
\fi
\usepackage{lmodern}
\ifPDFTeX\else
  % xetex/luatex font selection
\fi
% Use upquote if available, for straight quotes in verbatim environments
\IfFileExists{upquote.sty}{\usepackage{upquote}}{}
\IfFileExists{microtype.sty}{% use microtype if available
  \usepackage[]{microtype}
  \UseMicrotypeSet[protrusion]{basicmath} % disable protrusion for tt fonts
}{}
\makeatletter
\@ifundefined{KOMAClassName}{% if non-KOMA class
  \IfFileExists{parskip.sty}{%
    \usepackage{parskip}
  }{% else
    \setlength{\parindent}{0pt}
    \setlength{\parskip}{6pt plus 2pt minus 1pt}}
}{% if KOMA class
  \KOMAoptions{parskip=half}}
\makeatother
% Make \paragraph and \subparagraph free-standing
\makeatletter
\ifx\paragraph\undefined\else
  \let\oldparagraph\paragraph
  \renewcommand{\paragraph}{
    \@ifstar
      \xxxParagraphStar
      \xxxParagraphNoStar
  }
  \newcommand{\xxxParagraphStar}[1]{\oldparagraph*{#1}\mbox{}}
  \newcommand{\xxxParagraphNoStar}[1]{\oldparagraph{#1}\mbox{}}
\fi
\ifx\subparagraph\undefined\else
  \let\oldsubparagraph\subparagraph
  \renewcommand{\subparagraph}{
    \@ifstar
      \xxxSubParagraphStar
      \xxxSubParagraphNoStar
  }
  \newcommand{\xxxSubParagraphStar}[1]{\oldsubparagraph*{#1}\mbox{}}
  \newcommand{\xxxSubParagraphNoStar}[1]{\oldsubparagraph{#1}\mbox{}}
\fi
\makeatother

\usepackage{color}
\usepackage{fancyvrb}
\newcommand{\VerbBar}{|}
\newcommand{\VERB}{\Verb[commandchars=\\\{\}]}
\DefineVerbatimEnvironment{Highlighting}{Verbatim}{commandchars=\\\{\}}
% Add ',fontsize=\small' for more characters per line
\usepackage{framed}
\definecolor{shadecolor}{RGB}{241,243,245}
\newenvironment{Shaded}{\begin{snugshade}}{\end{snugshade}}
\newcommand{\AlertTok}[1]{\textcolor[rgb]{0.68,0.00,0.00}{#1}}
\newcommand{\AnnotationTok}[1]{\textcolor[rgb]{0.37,0.37,0.37}{#1}}
\newcommand{\AttributeTok}[1]{\textcolor[rgb]{0.40,0.45,0.13}{#1}}
\newcommand{\BaseNTok}[1]{\textcolor[rgb]{0.68,0.00,0.00}{#1}}
\newcommand{\BuiltInTok}[1]{\textcolor[rgb]{0.00,0.23,0.31}{#1}}
\newcommand{\CharTok}[1]{\textcolor[rgb]{0.13,0.47,0.30}{#1}}
\newcommand{\CommentTok}[1]{\textcolor[rgb]{0.37,0.37,0.37}{#1}}
\newcommand{\CommentVarTok}[1]{\textcolor[rgb]{0.37,0.37,0.37}{\textit{#1}}}
\newcommand{\ConstantTok}[1]{\textcolor[rgb]{0.56,0.35,0.01}{#1}}
\newcommand{\ControlFlowTok}[1]{\textcolor[rgb]{0.00,0.23,0.31}{\textbf{#1}}}
\newcommand{\DataTypeTok}[1]{\textcolor[rgb]{0.68,0.00,0.00}{#1}}
\newcommand{\DecValTok}[1]{\textcolor[rgb]{0.68,0.00,0.00}{#1}}
\newcommand{\DocumentationTok}[1]{\textcolor[rgb]{0.37,0.37,0.37}{\textit{#1}}}
\newcommand{\ErrorTok}[1]{\textcolor[rgb]{0.68,0.00,0.00}{#1}}
\newcommand{\ExtensionTok}[1]{\textcolor[rgb]{0.00,0.23,0.31}{#1}}
\newcommand{\FloatTok}[1]{\textcolor[rgb]{0.68,0.00,0.00}{#1}}
\newcommand{\FunctionTok}[1]{\textcolor[rgb]{0.28,0.35,0.67}{#1}}
\newcommand{\ImportTok}[1]{\textcolor[rgb]{0.00,0.46,0.62}{#1}}
\newcommand{\InformationTok}[1]{\textcolor[rgb]{0.37,0.37,0.37}{#1}}
\newcommand{\KeywordTok}[1]{\textcolor[rgb]{0.00,0.23,0.31}{\textbf{#1}}}
\newcommand{\NormalTok}[1]{\textcolor[rgb]{0.00,0.23,0.31}{#1}}
\newcommand{\OperatorTok}[1]{\textcolor[rgb]{0.37,0.37,0.37}{#1}}
\newcommand{\OtherTok}[1]{\textcolor[rgb]{0.00,0.23,0.31}{#1}}
\newcommand{\PreprocessorTok}[1]{\textcolor[rgb]{0.68,0.00,0.00}{#1}}
\newcommand{\RegionMarkerTok}[1]{\textcolor[rgb]{0.00,0.23,0.31}{#1}}
\newcommand{\SpecialCharTok}[1]{\textcolor[rgb]{0.37,0.37,0.37}{#1}}
\newcommand{\SpecialStringTok}[1]{\textcolor[rgb]{0.13,0.47,0.30}{#1}}
\newcommand{\StringTok}[1]{\textcolor[rgb]{0.13,0.47,0.30}{#1}}
\newcommand{\VariableTok}[1]{\textcolor[rgb]{0.07,0.07,0.07}{#1}}
\newcommand{\VerbatimStringTok}[1]{\textcolor[rgb]{0.13,0.47,0.30}{#1}}
\newcommand{\WarningTok}[1]{\textcolor[rgb]{0.37,0.37,0.37}{\textit{#1}}}

\usepackage{longtable,booktabs,array}
\usepackage{calc} % for calculating minipage widths
% Correct order of tables after \paragraph or \subparagraph
\usepackage{etoolbox}
\makeatletter
\patchcmd\longtable{\par}{\if@noskipsec\mbox{}\fi\par}{}{}
\makeatother
% Allow footnotes in longtable head/foot
\IfFileExists{footnotehyper.sty}{\usepackage{footnotehyper}}{\usepackage{footnote}}
\makesavenoteenv{longtable}
\usepackage{graphicx}
\makeatletter
\newsavebox\pandoc@box
\newcommand*\pandocbounded[1]{% scales image to fit in text height/width
  \sbox\pandoc@box{#1}%
  \Gscale@div\@tempa{\textheight}{\dimexpr\ht\pandoc@box+\dp\pandoc@box\relax}%
  \Gscale@div\@tempb{\linewidth}{\wd\pandoc@box}%
  \ifdim\@tempb\p@<\@tempa\p@\let\@tempa\@tempb\fi% select the smaller of both
  \ifdim\@tempa\p@<\p@\scalebox{\@tempa}{\usebox\pandoc@box}%
  \else\usebox{\pandoc@box}%
  \fi%
}
% Set default figure placement to htbp
\def\fps@figure{htbp}
\makeatother





\setlength{\emergencystretch}{3em} % prevent overfull lines

\providecommand{\tightlist}{%
  \setlength{\itemsep}{0pt}\setlength{\parskip}{0pt}}



 


\KOMAoption{captions}{tableheading}
\makeatletter
\@ifpackageloaded{caption}{}{\usepackage{caption}}
\AtBeginDocument{%
\ifdefined\contentsname
  \renewcommand*\contentsname{Table of contents}
\else
  \newcommand\contentsname{Table of contents}
\fi
\ifdefined\listfigurename
  \renewcommand*\listfigurename{List of Figures}
\else
  \newcommand\listfigurename{List of Figures}
\fi
\ifdefined\listtablename
  \renewcommand*\listtablename{List of Tables}
\else
  \newcommand\listtablename{List of Tables}
\fi
\ifdefined\figurename
  \renewcommand*\figurename{Figure}
\else
  \newcommand\figurename{Figure}
\fi
\ifdefined\tablename
  \renewcommand*\tablename{Table}
\else
  \newcommand\tablename{Table}
\fi
}
\@ifpackageloaded{float}{}{\usepackage{float}}
\floatstyle{ruled}
\@ifundefined{c@chapter}{\newfloat{codelisting}{h}{lop}}{\newfloat{codelisting}{h}{lop}[chapter]}
\floatname{codelisting}{Listing}
\newcommand*\listoflistings{\listof{codelisting}{List of Listings}}
\makeatother
\makeatletter
\makeatother
\makeatletter
\@ifpackageloaded{caption}{}{\usepackage{caption}}
\@ifpackageloaded{subcaption}{}{\usepackage{subcaption}}
\makeatother
\usepackage{bookmark}
\IfFileExists{xurl.sty}{\usepackage{xurl}}{} % add URL line breaks if available
\urlstyle{same}
\hypersetup{
  pdftitle={Class 12 RNASeq analysis},
  pdfauthor={Nathan Phan (PID: A17395036)},
  colorlinks=true,
  linkcolor={blue},
  filecolor={Maroon},
  citecolor={Blue},
  urlcolor={Blue},
  pdfcreator={LaTeX via pandoc}}


\title{Class 12 RNASeq analysis}
\author{Nathan Phan (PID: A17395036)}
\date{}
\begin{document}
\maketitle

\renewcommand*\contentsname{Table of contents}
{
\hypersetup{linkcolor=}
\setcounter{tocdepth}{3}
\tableofcontents
}

\subsection{Background}\label{background}

Today we will analysis some RNASeq data from Himes et al.~on the effects
of a common steroid (Dexamehasone) on airway smooth muscle cels (ASM
cells)

Our starting point is the ``counts'' data and ``metadata'' that contain
the count values for each gene in their different experiments (i.e.~cell
lines with or without the drug)

\subsection{Data Import}\label{data-import}

\begin{Shaded}
\begin{Highlighting}[]
\CommentTok{\# Complete the missing code}
\NormalTok{counts }\OtherTok{\textless{}{-}} \FunctionTok{read.csv}\NormalTok{(}\StringTok{"airway\_scaledcounts.csv"}\NormalTok{, }\AttributeTok{row.names=}\DecValTok{1}\NormalTok{)}
\NormalTok{metadata }\OtherTok{\textless{}{-}} \FunctionTok{read.csv}\NormalTok{(}\StringTok{"airway\_metadata.csv"}\NormalTok{, }\AttributeTok{row.names =} \DecValTok{1}\NormalTok{)}
\end{Highlighting}
\end{Shaded}

Let's have a wee peak at these objects:

\begin{Shaded}
\begin{Highlighting}[]
\FunctionTok{head}\NormalTok{(counts)}
\end{Highlighting}
\end{Shaded}

\begin{verbatim}
                SRR1039508 SRR1039509 SRR1039512 SRR1039513 SRR1039516
ENSG00000000003        723        486        904        445       1170
ENSG00000000005          0          0          0          0          0
ENSG00000000419        467        523        616        371        582
ENSG00000000457        347        258        364        237        318
ENSG00000000460         96         81         73         66        118
ENSG00000000938          0          0          1          0          2
                SRR1039517 SRR1039520 SRR1039521
ENSG00000000003       1097        806        604
ENSG00000000005          0          0          0
ENSG00000000419        781        417        509
ENSG00000000457        447        330        324
ENSG00000000460         94        102         74
ENSG00000000938          0          0          0
\end{verbatim}

\begin{quote}
Q1. How many genes are in this dataset?
\end{quote}

\begin{Shaded}
\begin{Highlighting}[]
\FunctionTok{nrow}\NormalTok{(counts)}
\end{Highlighting}
\end{Shaded}

\begin{verbatim}
[1] 38694
\end{verbatim}

\begin{quote}
Q. How many different experiments (columns in counts or rows in
metadata) are there?
\end{quote}

\begin{Shaded}
\begin{Highlighting}[]
\FunctionTok{ncol}\NormalTok{(counts)}
\end{Highlighting}
\end{Shaded}

\begin{verbatim}
[1] 8
\end{verbatim}

\begin{Shaded}
\begin{Highlighting}[]
\NormalTok{metadata}
\end{Highlighting}
\end{Shaded}

\begin{verbatim}
               dex celltype     geo_id
SRR1039508 control   N61311 GSM1275862
SRR1039509 treated   N61311 GSM1275863
SRR1039512 control  N052611 GSM1275866
SRR1039513 treated  N052611 GSM1275867
SRR1039516 control  N080611 GSM1275870
SRR1039517 treated  N080611 GSM1275871
SRR1039520 control  N061011 GSM1275874
SRR1039521 treated  N061011 GSM1275875
\end{verbatim}

\begin{Shaded}
\begin{Highlighting}[]
\FunctionTok{nrow}\NormalTok{(metadata)}
\end{Highlighting}
\end{Shaded}

\begin{verbatim}
[1] 8
\end{verbatim}

\begin{quote}
Q2. How many `control' cell lines do we have?
\end{quote}

\begin{Shaded}
\begin{Highlighting}[]
\FunctionTok{sum}\NormalTok{(metadata}\SpecialCharTok{$}\NormalTok{dex }\SpecialCharTok{==} \StringTok{"control"}\NormalTok{)}
\end{Highlighting}
\end{Shaded}

\begin{verbatim}
[1] 4
\end{verbatim}

\subsection{Toy differential gene
expression}\label{toy-differential-gene-expression}

To start our analysis let's calculate the mean counts for all genes in
the ``control'' experiments.

\begin{enumerate}
\def\labelenumi{\arabic{enumi}.}
\tightlist
\item
  Extract all ``control'' columns from the \texttt{counts} object
\item
  Calculate the mean for all rows (i.e.~genes) of these ``control''
  columns
\end{enumerate}

3-4. Do the same for ``treated'' 5. Compare these \texttt{control.mean}
and \texttt{treated.mean} values

\begin{Shaded}
\begin{Highlighting}[]
\NormalTok{control.inds }\OtherTok{\textless{}{-}}\NormalTok{ metadata}\SpecialCharTok{$}\NormalTok{dex }\SpecialCharTok{==} \StringTok{"control"}
\NormalTok{control.counts }\OtherTok{\textless{}{-}}\NormalTok{ counts[ , control.inds]}
\end{Highlighting}
\end{Shaded}

\begin{Shaded}
\begin{Highlighting}[]
\NormalTok{control.means }\OtherTok{\textless{}{-}} \FunctionTok{rowMeans}\NormalTok{(control.counts)}
\end{Highlighting}
\end{Shaded}

\begin{Shaded}
\begin{Highlighting}[]
\NormalTok{treated.means }\OtherTok{\textless{}{-}} \FunctionTok{rowMeans}\NormalTok{(counts[ , metadata}\SpecialCharTok{$}\NormalTok{dex }\SpecialCharTok{==} \StringTok{"treated"}\NormalTok{])}
\end{Highlighting}
\end{Shaded}

Store these together for ease of bookkeeping as \texttt{meancounts}

\begin{Shaded}
\begin{Highlighting}[]
\NormalTok{meancounts }\OtherTok{\textless{}{-}} \FunctionTok{data.frame}\NormalTok{(control.means, treated.means)}
\FunctionTok{head}\NormalTok{(meancounts)}
\end{Highlighting}
\end{Shaded}

\begin{verbatim}
                control.means treated.means
ENSG00000000003        900.75        658.00
ENSG00000000005          0.00          0.00
ENSG00000000419        520.50        546.00
ENSG00000000457        339.75        316.50
ENSG00000000460         97.25         78.75
ENSG00000000938          0.75          0.00
\end{verbatim}

\begin{quote}
Q5 (a). Create a scatter plot showing the mean of the treated samples
against the mean of the control samples. Your plot should look something
like the following.
\end{quote}

Make a plot of control vs treated mean values for all genes

\begin{Shaded}
\begin{Highlighting}[]
\FunctionTok{plot}\NormalTok{(meancounts)}
\end{Highlighting}
\end{Shaded}

\pandocbounded{\includegraphics[keepaspectratio]{Class-12---nathan-phan_files/figure-pdf/unnamed-chunk-12-1.pdf}}

\begin{quote}
Q6. Try plotting both axes on a log scale. What is the argument to
plot() that allows you to do this?
\end{quote}

\begin{Shaded}
\begin{Highlighting}[]
\FunctionTok{plot}\NormalTok{(meancounts, }\AttributeTok{log =} \StringTok{"xy"}\NormalTok{)}
\end{Highlighting}
\end{Shaded}

\begin{verbatim}
Warning in xy.coords(x, y, xlabel, ylabel, log): 15032 x values <= 0 omitted
from logarithmic plot
\end{verbatim}

\begin{verbatim}
Warning in xy.coords(x, y, xlabel, ylabel, log): 15281 y values <= 0 omitted
from logarithmic plot
\end{verbatim}

\pandocbounded{\includegraphics[keepaspectratio]{Class-12---nathan-phan_files/figure-pdf/unnamed-chunk-13-1.pdf}}

We often talk metrics like ``log2 fold-change''

\begin{Shaded}
\begin{Highlighting}[]
\CommentTok{\# control/treated}
\FunctionTok{log2}\NormalTok{(}\DecValTok{10}\SpecialCharTok{/}\DecValTok{10}\NormalTok{)}
\end{Highlighting}
\end{Shaded}

\begin{verbatim}
[1] 0
\end{verbatim}

\begin{Shaded}
\begin{Highlighting}[]
\FunctionTok{log}\NormalTok{(}\DecValTok{10}\SpecialCharTok{/}\DecValTok{20}\NormalTok{)}
\end{Highlighting}
\end{Shaded}

\begin{verbatim}
[1] -0.6931472
\end{verbatim}

\begin{Shaded}
\begin{Highlighting}[]
\FunctionTok{log2}\NormalTok{(}\DecValTok{20}\SpecialCharTok{/}\DecValTok{10}\NormalTok{)}
\end{Highlighting}
\end{Shaded}

\begin{verbatim}
[1] 1
\end{verbatim}

\begin{Shaded}
\begin{Highlighting}[]
\FunctionTok{log2}\NormalTok{(}\DecValTok{10}\SpecialCharTok{/}\DecValTok{40}\NormalTok{)}
\end{Highlighting}
\end{Shaded}

\begin{verbatim}
[1] -2
\end{verbatim}

Let's calculate the log2 fold chainge for our treated over control mean
counts.

\begin{Shaded}
\begin{Highlighting}[]
\NormalTok{meancounts}\SpecialCharTok{$}\NormalTok{log2fc }\OtherTok{\textless{}{-}}
\FunctionTok{log2}\NormalTok{(meancounts}\SpecialCharTok{$}\NormalTok{treated.means }\SpecialCharTok{/} 
\NormalTok{  meancounts}\SpecialCharTok{$}\NormalTok{control.means)}
\end{Highlighting}
\end{Shaded}

\begin{Shaded}
\begin{Highlighting}[]
\FunctionTok{head}\NormalTok{(meancounts)}
\end{Highlighting}
\end{Shaded}

\begin{verbatim}
                control.means treated.means      log2fc
ENSG00000000003        900.75        658.00 -0.45303916
ENSG00000000005          0.00          0.00         NaN
ENSG00000000419        520.50        546.00  0.06900279
ENSG00000000457        339.75        316.50 -0.10226805
ENSG00000000460         97.25         78.75 -0.30441833
ENSG00000000938          0.75          0.00        -Inf
\end{verbatim}

A common ``rule of thumb'' is a log fold change cutoff of +2 and -2 to
call genes ``Up regulated'' or ``Down regulated''.

\begin{Shaded}
\begin{Highlighting}[]
\FunctionTok{sum}\NormalTok{(meancounts}\SpecialCharTok{$}\NormalTok{log2fc }\SpecialCharTok{\textgreater{}} \SpecialCharTok{+}\DecValTok{2}\NormalTok{, }\AttributeTok{na.rm=}\NormalTok{T)}
\end{Highlighting}
\end{Shaded}

\begin{verbatim}
[1] 1846
\end{verbatim}

Number of ``down'' genes at -2 threshold

\begin{Shaded}
\begin{Highlighting}[]
\FunctionTok{sum}\NormalTok{(meancounts}\SpecialCharTok{$}\NormalTok{log2fc }\SpecialCharTok{\textgreater{}} \SpecialCharTok{{-}}\DecValTok{2}\NormalTok{, }\AttributeTok{na.rm=}\NormalTok{T)}
\end{Highlighting}
\end{Shaded}

\begin{verbatim}
[1] 22928
\end{verbatim}

\subsection{DESeq2 analysis}\label{deseq2-analysis}

Let's do this analysis properly and keep our inner stats nerd happy -
i.e.~are the differences we see between drug and no drug significant
given the replocate experiments.

\begin{Shaded}
\begin{Highlighting}[]
\FunctionTok{library}\NormalTok{(DESeq2)}
\end{Highlighting}
\end{Shaded}

\begin{verbatim}
Warning: package 'matrixStats' was built under R version 4.5.2
\end{verbatim}

For DESeq analysis we need three things

\begin{itemize}
\tightlist
\item
  count values (\texttt{countData})
\item
  metadata telling us about the columns in \texttt{countData}
  (\texttt{colData})
\item
  design of the experiment (i.e.~what do you want ot compare)
\end{itemize}

Our first function from DESeq2 will setup the input required for
analysis by storing all these 3 things faster.

\begin{Shaded}
\begin{Highlighting}[]
\NormalTok{dds }\OtherTok{\textless{}{-}} \FunctionTok{DESeqDataSetFromMatrix}\NormalTok{(}\AttributeTok{countData =}\NormalTok{ counts,}
                              \AttributeTok{colData =}\NormalTok{ metadata,}
                              \AttributeTok{design =} \SpecialCharTok{\textasciitilde{}}\NormalTok{dex)}
\end{Highlighting}
\end{Shaded}

\begin{verbatim}
converting counts to integer mode
\end{verbatim}

\begin{verbatim}
Warning in DESeqDataSet(se, design = design, ignoreRank): some variables in
design formula are characters, converting to factors
\end{verbatim}

The main function in DESeq2 that runs the analysis is called
\texttt{DESeq()}

\begin{Shaded}
\begin{Highlighting}[]
\NormalTok{dds }\OtherTok{\textless{}{-}} \FunctionTok{DESeq}\NormalTok{(dds)}
\end{Highlighting}
\end{Shaded}

\begin{verbatim}
estimating size factors
\end{verbatim}

\begin{verbatim}
estimating dispersions
\end{verbatim}

\begin{verbatim}
gene-wise dispersion estimates
\end{verbatim}

\begin{verbatim}
mean-dispersion relationship
\end{verbatim}

\begin{verbatim}
final dispersion estimates
\end{verbatim}

\begin{verbatim}
fitting model and testing
\end{verbatim}

\begin{Shaded}
\begin{Highlighting}[]
\NormalTok{res }\OtherTok{\textless{}{-}} \FunctionTok{results}\NormalTok{(dds)}
\FunctionTok{results}\NormalTok{(dds)}
\end{Highlighting}
\end{Shaded}

\begin{verbatim}
log2 fold change (MLE): dex treated vs control 
Wald test p-value: dex treated vs control 
DataFrame with 38694 rows and 6 columns
                 baseMean log2FoldChange     lfcSE      stat    pvalue
                <numeric>      <numeric> <numeric> <numeric> <numeric>
ENSG00000000003  747.1942     -0.3507030  0.168246 -2.084470 0.0371175
ENSG00000000005    0.0000             NA        NA        NA        NA
ENSG00000000419  520.1342      0.2061078  0.101059  2.039475 0.0414026
ENSG00000000457  322.6648      0.0245269  0.145145  0.168982 0.8658106
ENSG00000000460   87.6826     -0.1471420  0.257007 -0.572521 0.5669691
...                   ...            ...       ...       ...       ...
ENSG00000283115  0.000000             NA        NA        NA        NA
ENSG00000283116  0.000000             NA        NA        NA        NA
ENSG00000283119  0.000000             NA        NA        NA        NA
ENSG00000283120  0.974916      -0.668258   1.69456 -0.394354  0.693319
ENSG00000283123  0.000000             NA        NA        NA        NA
                     padj
                <numeric>
ENSG00000000003  0.163035
ENSG00000000005        NA
ENSG00000000419  0.176032
ENSG00000000457  0.961694
ENSG00000000460  0.815849
...                   ...
ENSG00000283115        NA
ENSG00000283116        NA
ENSG00000283119        NA
ENSG00000283120        NA
ENSG00000283123        NA
\end{verbatim}

\subsection{Volcano Plot}\label{volcano-plot}

This is a common summary result figure from these types of experiments
and plot the log2 fold-change vs teh adjusted p-value.

\begin{Shaded}
\begin{Highlighting}[]
\FunctionTok{plot}\NormalTok{(res}\SpecialCharTok{$}\NormalTok{log2FoldChange, )}
\end{Highlighting}
\end{Shaded}

\pandocbounded{\includegraphics[keepaspectratio]{Class-12---nathan-phan_files/figure-pdf/unnamed-chunk-26-1.pdf}}

\begin{Shaded}
\begin{Highlighting}[]
\FunctionTok{plot}\NormalTok{(res}\SpecialCharTok{$}\NormalTok{log2FoldChange, }\SpecialCharTok{{-}}\FunctionTok{log}\NormalTok{(res}\SpecialCharTok{$}\NormalTok{padj))}
\FunctionTok{abline}\NormalTok{(}\AttributeTok{v=}\FunctionTok{c}\NormalTok{(}\SpecialCharTok{{-}}\DecValTok{2}\NormalTok{,}\DecValTok{2}\NormalTok{), }\AttributeTok{col=}\StringTok{"red"}\NormalTok{)}
\FunctionTok{abline}\NormalTok{(}\AttributeTok{h=}\SpecialCharTok{{-}}\FunctionTok{log}\NormalTok{(}\FloatTok{0.04}\NormalTok{), }\AttributeTok{col=}\StringTok{"red"}\NormalTok{)}
\end{Highlighting}
\end{Shaded}

\pandocbounded{\includegraphics[keepaspectratio]{Class-12---nathan-phan_files/figure-pdf/unnamed-chunk-27-1.pdf}}

\subsection{Save our results}\label{save-our-results}

\begin{Shaded}
\begin{Highlighting}[]
\FunctionTok{write.csv}\NormalTok{(res, }\AttributeTok{file=}\StringTok{"my\_results.csv"}\NormalTok{)}
\end{Highlighting}
\end{Shaded}

\begin{Shaded}
\begin{Highlighting}[]
\FunctionTok{plot}\NormalTok{( res}\SpecialCharTok{$}\NormalTok{log2FoldChange,  }\SpecialCharTok{{-}}\FunctionTok{log}\NormalTok{(res}\SpecialCharTok{$}\NormalTok{padj), }
      \AttributeTok{xlab=}\StringTok{"Log2(FoldChange)"}\NormalTok{,}
      \AttributeTok{ylab=}\StringTok{"{-}Log(P{-}value)"}\NormalTok{)}
\end{Highlighting}
\end{Shaded}

\pandocbounded{\includegraphics[keepaspectratio]{Class-12---nathan-phan_files/figure-pdf/unnamed-chunk-29-1.pdf}}

\begin{Shaded}
\begin{Highlighting}[]
\CommentTok{\# Setup our custom point color vector }
\NormalTok{mycols }\OtherTok{\textless{}{-}} \FunctionTok{rep}\NormalTok{(}\StringTok{"gray"}\NormalTok{, }\FunctionTok{nrow}\NormalTok{(res))}
\NormalTok{mycols[ }\FunctionTok{abs}\NormalTok{(res}\SpecialCharTok{$}\NormalTok{log2FoldChange) }\SpecialCharTok{\textgreater{}} \DecValTok{2}\NormalTok{ ]  }\OtherTok{\textless{}{-}} \StringTok{"red"} 

\NormalTok{inds }\OtherTok{\textless{}{-}}\NormalTok{ (res}\SpecialCharTok{$}\NormalTok{padj }\SpecialCharTok{\textless{}} \FloatTok{0.01}\NormalTok{) }\SpecialCharTok{\&}\NormalTok{ (}\FunctionTok{abs}\NormalTok{(res}\SpecialCharTok{$}\NormalTok{log2FoldChange) }\SpecialCharTok{\textgreater{}} \DecValTok{2}\NormalTok{ )}
\NormalTok{mycols[ inds ] }\OtherTok{\textless{}{-}} \StringTok{"blue"}

\CommentTok{\# Volcano plot with custom colors }
\FunctionTok{plot}\NormalTok{( res}\SpecialCharTok{$}\NormalTok{log2FoldChange,  }\SpecialCharTok{{-}}\FunctionTok{log}\NormalTok{(res}\SpecialCharTok{$}\NormalTok{padj), }
 \AttributeTok{col=}\NormalTok{mycols, }\AttributeTok{ylab=}\StringTok{"{-}Log(P{-}value)"}\NormalTok{, }\AttributeTok{xlab=}\StringTok{"Log2(FoldChange)"}\NormalTok{ )}

\CommentTok{\# Cut{-}off lines}
\FunctionTok{abline}\NormalTok{(}\AttributeTok{v=}\FunctionTok{c}\NormalTok{(}\SpecialCharTok{{-}}\DecValTok{2}\NormalTok{,}\DecValTok{2}\NormalTok{), }\AttributeTok{col=}\StringTok{"gray"}\NormalTok{, }\AttributeTok{lty=}\DecValTok{2}\NormalTok{)}
\FunctionTok{abline}\NormalTok{(}\AttributeTok{h=}\SpecialCharTok{{-}}\FunctionTok{log}\NormalTok{(}\FloatTok{0.1}\NormalTok{), }\AttributeTok{col=}\StringTok{"gray"}\NormalTok{, }\AttributeTok{lty=}\DecValTok{2}\NormalTok{)}
\end{Highlighting}
\end{Shaded}

\pandocbounded{\includegraphics[keepaspectratio]{Class-12---nathan-phan_files/figure-pdf/unnamed-chunk-30-1.pdf}}




\end{document}
